\documentclass{article}
\usepackage{amsmath, amsthm}
\usepackage{hyperref}
\hypersetup{
  allcolors=green
}


\newtheorem{theorem}{Theorem}[section]
\newtheorem{proposition}[theorem]{Proposition}
\newtheorem{definition}[theorem]{Definition}
\newtheorem{example}[theorem]{Example}
\newtheorem{lemma}[theorem]{Lemma}
\title{Notes on Smooth Manifolds}
\author{Job Hernandez Lara}
\date{}

\begin{document}

\maketitle
\tableofcontents

\section{Introduction}
I am writing these notes to improve my understanding of this beautiful field.
These notes are based on ``Introduction to Smooth Manifolds'' by John Lee and ``Topology'' by James Munkres.

In one of his blog posts, Terrence Tao claims that to understand mathematics one needs to connect pre-rigorous understanding, i.e., intuitive computational understanding to theoretical or proof based understanding. So, throughout the notes, I will be connecting the two levels.

\section{Topology}

\subsection{Topological Spaces}

A topological space is defined as a pair consisting of a set $ X $ and a topology $ T $ on $ X $. Here, $ T $ simply means a collection of open subsets of $ X $.

This collection has the following properties:

\begin{enumerate}
\item The sets $ \emptyset $ and $ X $ are in $ T $ so they are open.
\item Given any subcollection of $ T $ the union is in $ T $ so it is open.
\item Given a finite subcollection in $ T $ the intersection is in $ T $ so it is open.
\end{enumerate}

Let $ U $ be in $ X $. $ U $ is an open set of $ X $ if it is in $ T $.
\subsection{Hausdorff Spaces}
A topological space is Hausdorff if there exists neighborhoods $ U_{1}, U_{2} $ of any two distinct pairs $ x_{1}, x_{2} $ that are disjoint.

If two sets are disjoint then $ A \cap B = \emptyset $, i.e., the sets do not share any elements in common.
\subsection{Basis and Countability}

Let $ X $ be a set. Then a basis $ B $ of $ X $ is a collection of subsets of $ X $ with the following
properties.

\begin{enumerate}
\item Let $ x \in X $ and $ S \in B $. Then $ x \in B $.
\item Let $ B_{1} \in B $. Let $ B_{2} \in B $. Then there exists  $ x \in B_{3} $ such that $ B_{3} \subset B_{1} \cup B_{2} $.
\end{enumerate}

We then can use $ B $ to generate a topology $ T $ as follows.

Let $ U \subset X $. Then $ U $ is open, i.e., $ U \in T $. Then  $ x \in B $ and $ B \subset U $ for any basis element in $ B $.

\subsection{Subspaces}

We can construct new topological spaces by taking subsets of existing ones, i.e., subspaces.

Let $ X $ be a topological space and $ S \subseteq X $ then a subspace topology on $ S $ is defined
by letting  $ U \subseteq  S $. Then the following statements are true.
%\subsubsection{Properties of subspaces}

\begin{enumerate}
\item If $ U $ is open in $ S $ then for a open subset of $ V \subseteq X $, $ U =  V \cup  S $.
\item If for an open subset of $ V \subseteq X $, $ U =  V \cup  S $ then  $ U $ is open in $ S $.
\end{enumerate}

\section{Topological Manifolds}

Topological Manifolds are topological spaces with a given structure that makes behave like Euclidian space i.e., $ R^n $.

Essentially, this structure will allow calculus on smooth manifolds because it gives it a coordinate system.

If a topological space $ X $ is a topological manifold then the following statements are true.

\begin{enumerate}
\item $ X $ is a Hausdorff space.
\item $ X $ is second countable.
\item $ X $ looks locally like Euclidian.
\end{enumerate}

If a space $ X $ is second countable then $ X $ has a countable basis. If a space $ X $ has a countable basis then there exists a set $ B $ of neighborhood basis $ U_{i}, U_{i+1}, U_{n} $ for each point $ p \in U_{n} $ and each $ U_{n} $ contains at least one element of $ B $.

Here, `` $ U $ is neighborhood of x`` means that ``$ U $ is an open set containing $ x $.

\subsection{Coordinate charts}
A coordinate chart, on a topological manifold $ M $, is a pair $ U $ and $ \varphi $ where $ U $ is an open subset of $ M $ and $ \varphi: U \rightarrow \hat{U} $ is a homeomorphism.

If $ \varphi $ is a homeomorphism then $  \varphi^{-1}: \hat{U} \rightarrow U $. $ \varphi $ is a map from a region of $ M $ to  $ \hat{U} = \varphi(U) \subseteq R^n $ and each point $ p \in M $ is in the chart such that $ \varphi(p) \rightarrow \hat{U} $.

%\subsection{Properties of Topological Manifolds}
\begin{lemma}
Let $ M $ be any topological manifold. Then $ M $ has a countable basis of precompact coordinate balls.
\end{lemma}

\subsubsection{Connectivity}
\begin{proposition}
  Let $ M $ be a topological manifold.  Then the following statements are true based on lemma 3.1.
  \begin{enumerate}
  \item $ M $ is locally path connected.
  \item If $ M $ is connected then $ M $ is locally path connected and if $ M $ is locally path connected then $ M $ is connected.
  \item Let $ S $ be the components of $ M $ and let $ T $ be the path components of $ M $. The $ S $ and $ T $ are the same.
  \item Let $ U $ be a component of $ M $. Then $ U $ is an open subset of $ M $ and $ U $ is connected topological manifold.
  \end{enumerate}
\end{proposition}

\begin{definition}[connected]
Let $ X_{1} $ and $  X_{2} $ be any two disjoint, non empty,  open subsets of a topological space $ X $. If no such subsets exist whose union is $ X $ then $ X $ is connected.
\end{definition}
\begin{definition}[path connected]
  Let $ x_{1}, x_{2} $ be any pair in $ X $ that can be joined in a path in $ X $. Then $ X $ is path connected.
\end{definition}
\begin{definition}[locally path connected]
  If a topological space $ X $ has basis of path connected open subsets then $ X $ is locally path connected.
\end{definition}

\subsubsection{Local Compactness and Paracompactness}
\begin{proposition}
  Let $ X $ be any topological manifold. Then $ X $ is locally compact.
\end{proposition}

\begin{proof}
  See proof of lemma 3.1 in ``Introduction of Smooth Manifolds''.
\end{proof}

  %\begin{proposition}
%\end{proposition}

%\begin{proof}
%This follows directly from the properties of inequalities in the real numbers.
%\end{proof}

%\subsection{Main Theorem}

%\begin{theorem}

%\end{theorem}

%\begin{proof}

%\end{proof}

%\section{Examples}

%\begin{example}

%\end{example}

\end{document}
